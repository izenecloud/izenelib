\hypertarget{ClockTimer_8h}{
\section{/home/Kevin/izenelib/include/util/ClockTimer.h File Reference}
\label{ClockTimer_8h}\index{/home/Kevin/izenelib/include/util/ClockTimer.h@{/home/Kevin/izenelib/include/util/ClockTimer.h}}
}
Timer using wall clock instead of CPU ticks.  


{\tt \#include $<$boost/date\_\-time/posix\_\-time/posix\_\-time.hpp$>$}\par
\subsection*{Classes}
\begin{CompactItemize}
\item 
class \hyperlink{classizenelib_1_1util_1_1ClockTimer}{izenelib::util::ClockTimer}
\begin{CompactList}\small\item\em A timer object similar to {\tt boost::timer} but measures total elapsed time instead of CPU time. \item\end{CompactList}\end{CompactItemize}


\subsection{Detailed Description}
Timer using wall clock instead of CPU ticks. 

\begin{Desc}
\item[Author:]Ian Yang \end{Desc}
\begin{Desc}
\item[Date:]Created $<$2009-05-06 09:34:37$>$ 

Updated $<$2009-05-26 10:35:24$>$ In some performance measurement, the total elapsed time is more important than just process or CPU time. {\tt \hyperlink{classizenelib_1_1util_1_1ClockTimer}{izenelib::util::ClockTimer}} is such a tool recording the total elapsed time.\end{Desc}
Because it use the Boost Date Time library, you may need to link that library. 

Definition in file \hyperlink{ClockTimer_8h-source}{ClockTimer.h}.